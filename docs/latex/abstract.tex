\chapter*{Resumo}
%\addcontentsline{toc}{chapter}{Resumo}

A Ciência do Clima Espacial é um campo de pesquisa vital que visa compreender as condições na superfície do Sol, que podem afetar negativamente a vida na Terra. Apesar de ser um campo bem desenvolvido, as condições que levam a essas fenómenos nefastos ainda não são totalmente compreendidas. Esta limitação é principalmente atribuída à dificuldade em obter dados de alta qualidade da superfície do Sol.

De forma a contornar este problema, alguns modelos de simulação tentam extrapolar as condições no Sol analisando dados de outras medições. Um exemplo disso é o MULTI-VP que usa magnetogramas de várias fontes (por exemplo, Wilcox Solar Observatory) e determina a estrutura do campo magnético de fundo do vento solar. No entanto, essas simulações demoram muito tempo a convergir e requerem estimativas iniciais de especialistas, feitas manualmente. Recentemente, uma abordagem baseada em Machine Learning foi projetada para atenuar esses problemas. Esta removeu a necessidade de estimativas iniciais, prevendo automaticamente as condições iniciais da simulação. Além disso, provou que pode haver uma redução significativa no tempo de execução do simulador. Apesar disso, os modelos de previsão ainda não são robustos o suficiente para serem usados em aplicações do mundo real. Como em muitos outros problemas de Machine Learning, acreditamos que a presença de anomalias no conjunto de dados de treino tenha prejudicado a sua performance.

Uma possível teoria é que as condições iniciais estão diretamente correlacionadas com o tempo de computação da simulação, e que melhores estimativas inicias levam a tempos de execução do MULTI-VP mais rápidas. Posto isto, nesta dissertação, aplicamos vários métodos de clustering e de deteção de anomalias de forma a melhorar a qualidade das condições iniciais e verificar se isso resultava em simulações mais rápidas do MULTI-VP.

Vários métodos de clustering foram testados nos dados de magnetogramas usados no treino do modelo de previsão, para determinar qual seria o mais apropriado. Adicionalmente, um conjunto de métodos de treino baseados nas técnicas de clustering foram testados, dos quais pelo menos um gerou estimativas mais próximas às previsões da simulação. Apesar disto, não houve qualquer redução no tempo de execução da simulação.

Para melhorar os resultados dos experimentos anteriores, vários métodos de detecção de anomalia adversária foram testados. Os modelos foram retreinados sem as anomalias detectadas o que resultou em piores condições de fluxo inicial quando
comparado ao resultado final do MULTI-VP; no entanto, o tempo de computação foi ligeiramente menor do que na implementação anterior.

Concluindo, os resultados das experiências com os métodos de clustering e de deteção de anomalias adversariais parecem indicar que o desempenho do simulador não está correlacionado com a proximidade das condições iniciais às soluções da simulação.

% Com o fim de melhorar os resultados das experiências anteriores, vários métodos de aprendizagem adversarial foram implementados e testados. Os modelos de previsão treinados com os métodos de clustering sofreram um novo processo de treino sem os dados anomalos no dataset de treino. Isto resultou em piores condições iniciais quando comparado com as estimativas finais da simulação. Contudo, a simulação demorou menos tempo a alcançar um solução viável do que com as condições iniciais dos modelos anteriores.

% Podemos classificar os outliers como dados fora da distribuição, anomalias, novidades e desvios. Estes podem ser causados por falhas de sistema, ruído de sensor, erros humanos e operações maliciosas. Mais recentemente, GANs têm provado ser métodos eficazes de deteção de anomalias de maneiras não supervisionadas.

% Com isto, esta dissertação propõe uma abordagem baseada em GANs para deteção de outliers em perfis de vento solar que depois serão usados para treinar um modelo de previsão. O impacto do novo método será avaliado em relação aos resultados obtidos através de um modelo de previsão treinado com dados não normalizados.

% Vários métodos de deteção de anomalias baseados em GANs foram estudados para determinar qual se adequava melhor ao problema. Com as lições aprendidas neste passo, propusemos uma metodologia que incorpora GANs para deteção de amostras de perfis de vento solar anómalos nos dados usados para treinar modelos para estimativa de condições iniciais. Um método de treino foi projectado, juntamente com um mecanismo de detecção de anomalias. Os critérios de avaliação da metodologia proposta também foram definidos.
\vspace{1cm}
\textbf{Palavras-chave:} Meteorologia Espacial, Vento Solar, Aprendizagem Computacional, Clustering, Deteção Adversarial de Anomalias


\chapter*{Abstract}
%\addcontentsline{toc}{chapter}{Abstract}
Space Weather Science is a vital field of research that aims to understand the conditions on the Sun's surface, which can negatively impact life on Earth. Despite being a well-researched field, the conditions that lead to these phenomena are still not fully understood. This limitation is mainly attributed to the difficulty in acquiring high-quality data from the Sun's surface.

To circumvent this issue, some simulation models try to extrapolate the conditions on the Sun by analyzing data from other measurements. An example of this is MULTI-VP which uses magnetograms from various sources (\textit{e.g.}, Wilcox Solar Observatory) and determines the structure of the solar wind's background magnetic field. However, these simulations take a long time to converge and require initial expert estimations, which are handmade. Recently, a machine learning approach has been designed to attenuate these issues. It removed the need for initial estimates by automatically predicting the starting conditions of the simulation. In addition, it has shown that there can be a significant reduction in the execution time of the simulator. Despite this, given their lack of physical cohesion, the prediction models are still not robust enough for real-world applications. 

We posit that initial conditions directly influence the computation time of the simulation and that better initial estimates will lead to faster executions. Thus, in this dissertation, we applied clustering and anomaly detection techniques to improve the quality of the initial conditions and determine if this would lead to faster MULTI-VP executions.

Several clustering experiments were conducted on the available magnetogram dataset to determine the best-suited clustering method. In addition, various clustering-based approaches for enhancing the prediction model were tested, with the selected method producing initial flow conditions closer to the simulation outputs. Despite this, there was no reduction in the computation time of the simulation.

To improve the previous experiments' results, various adversarial anomaly detection methods were designed and tested. The prediction models of the clustering-based experiments were retrained without the detected anomalies and resulted in worse initial flow conditions when compared to the final output of MULTI-VP; however, this time, the computation time was slightly lower than on the previous implementation.

In conclusion, the experiments conducted in this dissertation seem to indicate that the performance of the MULTI-VP simulator is not directly linked to the initial flow condition's approximation to the final solutions.



% Thus, this dissertation aims to apply GANs to detect anomalies in solar wind profiles to measure
% their influence on the already developed prediction models. This will be done by taking
% advantage of the discriminative and generative abilities of GANs. In addition, we will also evaluate the impacts of the improved prediction models, trained with normalized datasets, on the performance of the simulation.

% Several GAN-based anomaly detection methods were studied to determine which suited the problem. With the insights from this step, we proposed a methodology incorporating GANs for detecting anomalous solar wind profile samples in the data used to train models for initial condition estimation. A training method was presented along with an anomaly detection mechanism. Additionally, the evaluation criteria for the proposed methodology were also defined.

% TODO conclusion
\vspace{1cm}
\noindent\textbf{Keywords:} Space Weather, Solar Wind, Machine Learning, Clustering, Adversarial Anomaly Detection