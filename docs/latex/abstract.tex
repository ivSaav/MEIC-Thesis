\chapter*{Resumo}
%\addcontentsline{toc}{chapter}{Resumo}

A Ciência do Clima Espacial é um campo de pesquisa vital que visa compreender as condições na superfície do Sol, que podem afetar negativamente a vida na Terra. Apesar de ser um campo bem desenvolvido, as condições que levam a essas fenómenos nefastos ainda não são totalmente compreendidas. Esta limitação é principalmente atribuída à dificuldade em obter dados de alta qualidade da superfície do Sol.

De forma a contornar este problema, alguns modelos de simulação tentam extrapolar as condições no Sol analisando dados de outras medições. Um exemplo disso é o MULTI-VP que usa magnetogramas de várias fontes (por exemplo, Wilcox Solar Observatory) e determina a estrutura do campo magnético de fundo. No entanto, essas simulações demoram muito tempo a convergir e requerem estimativas iniciais de especialistas, feitas manualmente. Recentemente, uma abordagem baseada em Machine Learning foi projetada para atenuar esses problemas. Esta removeu a necessidade de estimativas iniciais, prevendo automaticamente as condições iniciais da simulação. Além disso, provou que pode haver uma redução significativa no tempo de execução do simulador. Apesar disso, os modelos de previsão ainda não são robustos o suficiente para serem usados em aplicações do mundo real. Como em muitos outros problemas de Machine Learning, acreditamos que a presença de outliers no conjunto de dados de treinamento tenha prejudicado a sua performance.

Podemos classificar os outliers como dados fora da distribuição, anomalias, novidades e desvios. Estes podem ser causados por falhas de sistema, ruído de sensor, erros humanos e operações maliciosas. Mais recentemente, GANs têm provado ser métodos eficazes de deteção de anomalias de maneiras não supervisionadas.

Com isto, esta dissertação propõe uma abordagem baseada em GANs para deteção de outliers em perfis de vento solar que depois serão usados para treinar um modelo de previsão. O impacto do novo método será avaliado em relação aos resultados obtidos através de um modelo de previsão treinado com dados não normalizados.

Vários métodos de deteção de anomalias baseados em GANs foram estudados para determinar qual se adequava melhor ao problema. Com as lições aprendidas neste passo, propusemos uma metodologia que incorpora GANs para deteção de amostras de perfis de vento solar anómalos nos dados usados para treinar modelos para estimativa de condições iniciais. Um método de treino foi projectado, juntamente com um mecanismo de detecção de anomalias. Os critérios de avaliação da metodologia proposta também foram definidos.


\chapter*{Abstract}
%\addcontentsline{toc}{chapter}{Abstract}
Space Weather Science is a vital field of research that aims to understand the conditions on the Sun's surface, which can negatively impact life on Earth. Despite being a well-developed field, the conditions that lead to these nefarious phenomena are still not fully understood. This limitation is mainly attributed to the difficulty in acquiring high-quality data from the Sun's surface.

To circumvent this issue, some simulation models try to extrapolate the conditions on the
Sun by analyzing data from other measurements. An example of this is MULTI-VP which
uses magnetograms from various sources (e.g., Wilcox Solar Observatory) and determines the
structure of the background magnetic field. However, these simulations take a long time to converge and require initial expert estimations, which are done by hand. Recently, a machine learning approach has been designed to attenuate these issues. It removed the need for initial estimates by automatically predicting the starting conditions of the simulation. In addition, it has proved that there can be a significant reduction in the execution time of the simulator. Despite this, the prediction models are still not robust enough to be used in real-world applications. Like many other machine learning problems, we believe the presence of outliers in the training dataset hampers its predictive performance.

We can classify outliers as out-of-distribution data, anomalies, novelties, and deviations. These can be caused by system failures, sensor noise, human errors, and malicious operations. More recently, GANs have proved to be effective unsupervised anomaly detection methods.

Thus, this dissertation aims to apply GANs to detect anomalies in solar wind profiles to measure
their influence on the already developed prediction models. This will be done by taking
advantage of the discriminative and generative abilities of GANs. In addition, we will also evaluate the impacts of the improved prediction models, trained with normalized datasets, on the performance of the simulation.

Several GAN-based anomaly detection methods were studied to determine which suited the problem. With the insights from this step, we proposed a methodology incorporating GANs for detecting anomalous solar wind profile samples in the data used to train models for initial condition estimation. A training method was presented along with an anomaly detection mechanism. Additionally, the evaluation criteria for the proposed methodology were also defined.

% TODO conclusion
\vspace{1cm}
\noindent\textbf{Keywords:} Space Weather, Machine Learning, Anomaly Detection, Generative Adversarial Networks, Solar Wind