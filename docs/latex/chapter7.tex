\chapter{Clustering}\label{chap:clustering}

To better understand the data and to help with the task of finding anomalies, we will be using clustering techniques. These techniques are used to group data points that are similar to each other. The goal is to find groups of data points that are similar to each other, but different from the rest of the data. This is done by finding the distance between each data point and the rest of the data. The distance between two data points is calculated using a distance metric.

In this chapter, the clustering methods that were used will be briefly explained in section \ref{sec:clustering_methods}. Next, the dimensionality reduction methods used in some of the tests will be presented in section \ref{sec:dim_reduction}. 
% TODO

\section{Clustering Methods}\label{sec:clustering_methods}
Clustering is defined as the task of finding and then grouping the data values that are close to each other in some way. Because of this, clustering is not only a useful technique for data analysis but also for the task of anomaly detection, as abnormal samples tend to be far from the rest of the data. In this section, the clustering methods that were used in this work will be presented.

\subsection{K-Means}\label{sec:kmeans}

\subsection{SOM}\label{sec:som}

\subsection{Aglomerative Clustering}\label{sec:aglomerative}

\subsection{DBSCAN}\label{sec:dbscan}


\section{Dimensionality Reduction}\label{sec:dim_reduction}

\section{Experiments}\label{sec:clustering_experiments}

\section{Results}\label{sec:clustering_results}
