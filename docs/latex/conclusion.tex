\chapter{Conclusions} \label{chap:conclusion}

The necessity to consistently predict the Sun's conditions that lead to the most extreme events has become an increasingly important study. However, technological difficulties make it challenging to obtain real-time data from the Sun's surface. Multiple numeric simulators have tried to fill this gap by extrapolating these conditions based on limited observations from Earth. In this dissertation, we have explained the problems (sec. \ref{sec:prob_definition}) associated with these solutions that severely affect the ability to generate solar estimations promptly. These issues included the long execution time of the simulation models as well as the need for initial expert estimations. Additionally, it was posited that the use of machine learning techniques to predict the original conditions suffered greatly from anomalies in the training dataset.

To address these issues, we first studied the current most popular GAN-based techniques for anomaly detection. The state-of-the-art analysis (sec. \ref{sec:stoa_results}) showed that GAN-based anomaly detection techniques performed the task better than other mainstream methods. With this in mind, we proposed a GAN-based anomaly detection technique (sec. \ref{sec:method}) that detects anomalies in the training dataset used by the prediction models. In the same section, the evaluation methods that will be used for the validation of the results were also explained. Finally, the work plan (sec. \ref{sec:work_plan}) was presented, which included the tasks that will be performed to achieve the objectives of the dissertation.




