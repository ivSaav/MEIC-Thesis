\chapter{Hypothesis Discussion}\label{chap:hypothesis_disc}
Following the work carried out in this thesis, this chapter intends to evaluate the proposed hypothesis from Section \ref{sec:hypothesis}. In it, we posit that the computation time of MULTI-VP is directly linked to the quality of the estimates used to kickstart the simulation. From this assumption, it follows that using better initial estimates would lead to a faster simulation computation time. After an extensive analysis of the results, it can be concluded that this hypothesis does not hold.

The following answers to the research questions help explain this conclusion:

\begin{description}
    \item{RQ1} \textit{Are clustering methods capable of detecting unknown characteristics in the dataset that would assist the prediction task?} The clustering experiments' results show that using this approach, we could generate estimates closer to the simulation outputs. The models trained on the resulting KMeans clustering of the PCA of the input variables managed to produce better estimates than the single model trained on the whole dataset. This shows that by dividing the dataset into clusters of approximately the same size, we effectively attenuated the data dispersion issues from the original approach in Barros \cite{barros_InitialConditionEstimation_}.

    \item[RQ2] \textit{Do the estimates obtained with clustering methods significantly improve the simulation’s
performance?} Despite being able to generate better initial estimates than the ones provided by the experts and the baseline model, this approach didn't significantly reduce the computation time of the MULTI-VP simulation. The impact on the computation time was assessed by comparing the mean speedup of simulations obtained by feeding MULTI-VP simulation with predictions from both the baseline and the devised clustering models. The baseline model achieved a mean speedup of 1.06, which constituted a slight improvement over the expert estimates. On the other hand, the new approach with the clustering models resulted in a mean speedup of only 1.05, which means that there was a marginal increase in the computation time of the simulation when compared to the baseline approach.

    \item[RQ3] \textit{Can adversarial architectures accurately detect outliers in solar wind profiles?} The results from section \ref{sec:gan_experiments} show that it is possible to use adversarial detection methods in this type of data. Even tho the linear GAN and the AAE managed to clean the dataset, the number of detected normal profiles (FN) is still very high compared to MAD-GAN, which might indicate that the previous architectures are not very suited for this task. In addition, we showed that grouping consecutive profiles into windows and then using these on LSMT-based GAN architectures (as MAD-GAN) proved very effective, surpassing the other approaches.

    \item[RQ4] \textit{Does the resulting dataset significantly improve the predictive ability of the RNN?} Despite removing most anomalous data from the training dataset, the estimates' quality decreased compared to the previous clustering models. The results in the adversarial experiments showed that the predictions from the clustering models trained on the clean dataset were significantly worse than the previous approaches (in some cases, worse than the expert estimates). This might indicate that the removed anomalous profiles provided key features for the RNN training, and excluding them from this phase hindered the predictions' quality. However, it is still important to note that we are using a small portion of the dataset randomly selected by hand, which might not represent the entire dataset.

    \item[RQ5] \textit{Does the improved predictive ability of the RNN result in a further reduction of execution
time for MULTI-VP?} Even with worse estimates than with the previous clustering models, the new approach managed to obtain a better speedup (from 1.05 to 1.06). This disproves the central hypothesis of this thesis that initial partial flow estimates closer to the final simulation ones would reduce the computation time. During the experiments, it was noticed that the model was still predicting extreme values when given anomalous inputs, even without seeing anomalies in the training phase. Further work is needed in this step to be able to reach a better conclusion. A possible approach would be to exclude the anomalous profiles from the simulation to ensure that these are the cause for the increased computation times.
\end{description} 